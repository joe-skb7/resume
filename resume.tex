\documentclass[11pt,letterpaper,sans]{moderncv}
\moderncvtheme{classic}
\usepackage[scale=0.8]{geometry}

% Personal data
\firstname{Semen}
\familyname{Protsenko}
\title{Software Engineer}
\address{Austin, TX, 78745}{}
\mobile{+1 (512) 986-2526}
\email{joe.skb7@gmail.com}
\photo[64pt][0.4pt]{images/photo}
\social[linkedin]{sam-protsenko}
\social[github]{joe-skb7}

% Content
\begin{document}

\maketitle

\section{Summary}

Embedded software engineer with more than 10 years of professional experience.
Main area of expertise is Linux kernel, bootloaders, firmwares and system level
user space programming on ARM-based devices. For the last 5 years has been
contributing to open-source projects and been involved with upstreaming
activity. Has prior background in cross-platform applications development,
algorithms and electronics. Active member of open-source community, mentor,
speaker.

\section{Experience}
\cventry{2020--Present}{Software Engineer}{Linaro}{}{}
  {Team: Linaro Consumer Group, Kernel Team. \newline{}
   Role: Supporting the WinLink’s \textsc{E850-96} board. \newline{}
   Activities: Board bring up, enabling it in mainline kernel and U-Boot,
   upstreaming.}
\cventry{2018--2020}{Team Lead}{GlobalLogic}{Kyiv, Ukraine}{}
  {Project: Texas Instruments (Linaro assignee for LCG). \newline{}
   Role: Supporting the \textsc{BeagleBoard X15} board and upstreaming work.
   \begin{itemize}
     \item Porting Android to a new board and upstreaming it with a small team
     \item Adopting VPN from upstream kernel in AOSP
     \item Implementing Android 10 boot sequence
     \item Running Android 10 with mainline kernel on \textsc{BeagleBoard X15}
   \end{itemize}}
\cventry{2012--2018}{Software Engineer}{GlobalLogic}{Kyiv, Ukraine}{}
  {Projects: Texas Instruments (directly and as Linaro assignee for LEG and
   LMG), others. \newline{}
   Role: Platform development in Linux kernel and bootloaders, upstreaming.
   \begin{itemize}
     \item Supporting Android kernel on TI OMAP4/OMAP5 SoC
     \item Board bring ups, bug fixing, migrating to new kernel versions
     \item Implementing and supporting device drivers
     \item Maintaining boot sequence and fastboot in U-Boot
     \item Implementing XIP boot from NOR flash (automotive)
     \item System boot time optimizations (automotive)
   \end{itemize}
  }
\cventry{2009--2012}{Software Developer}{AlterEGO}{Kramatorsk, Ukraine}{}
  {Role: Cross-platform software design and coding (C++/Qt). \newline{}
   Activities: Implementing OCR, game clients, interaction with web, apps for
   automating diverse tasks.}

\section{Education}
\cventry{2008--2009}
  {Master of Science in Computer Aided Engineering}
  {\href{http://eng.dgma.donetsk.ua/}{\newline{}
   Donbass State Engineering Academy}}{Kramatorsk, Ukraine}{}{}
\cventry{2004--2008}
  {Bachelor of Science in Computer Engineering Technology}
  {\href{http://eng.dgma.donetsk.ua/}{\newline{}
   Donbass State Engineering Academy}}{Kramatorsk, Ukraine}{}{}

\pagebreak

\section{Computer skills}
\cvdoubleitem{Languages}{C, C++, Bash}{Projects}{Linux kernel, U-Boot, Android}
\cvdoubleitem{CPUs}{ARM32, AArch64, x86}{MCUs}{AM57x, AM65x, STM32, MSP430}
\cvdoubleitem{Tools}{GCC, GDB, Make, Git, Vim}{Hardware}{JTAG, scope, logical
                     analyzer}

\section{Languages}
\cvlanguage{Ukrainian}{Native or bilingual proficiency}{}
\cvlanguage{English}{Professional working proficiency}{}

\section{Other Activities}

\cvitem{2019}{Organizer of ``Root Linux Conference'' in Kyiv}
\cvitem{2018--2022}{Creator and instructor of Linux kernel course}
\cvitem{2017--2019}{Speaker at Linaro and GlobalLogic events}
\cvitem{2015--Today}{Active member on StackOverflow (>10k reputation)}
\cvitem{2011--2022}{Maintainer of ``INSTEAD'' Debian package}

\end{document}
